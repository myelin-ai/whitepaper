\section{Environment}

\subsection{Separation of Mind and Body}
In our only role model for General Intelligence, the human brain, a separation of the mind and
the body it is residing in is unfeasable~\cite{Dudai2014}. Because we assume that nature is an 
indicator for the necessary attributes of General Intelligence, the consequent thing to do is not 
abstract away the bodily environment and its steering needs that gave rise to the mind in the first place~\cite{Jekely2010}. 
A simulation capable of producing AGI must therefore in our opinion contain not only ``brains in jars'', but 
also provide a body with which a simulated organism can interact with other organisms in a social manner and a 
sandbox-like environment that is susceptible to change.
But how far should such a simulation go? Which attributes of the real world are worth simulating in a way 
that contributes to the emergence of AGI\@?

\subsection{Social Interaction}
The human brain does not only thrive on sensory input, it
\emph{demands} interaction. Deprived of these stimulations,
the brain suffers damage and enforces sensory feedback by employing
hallucinations~\cite{Grassian2006}. From this, we can infer that
our simulation must provide organisms with information about their surroundings.
This includes the ability to communicate with other organisms over some
kind of arbitrary protocol. By emiting own messages over the protocol and
rewarding organisms that adhere to instructions broadcasted in this manner,
we can steer how the organisms interpret and shape this common ``language''.

\subsection{Life-Sustaining resources}
In order to encourage neuroplasticity and adaptivity, the organisms should be 
forced out of their comfort zone, as they could stagnate otherwise.
To achieve this, the environment continuously alters itself to shift its resources.
Another great booster of evolution is competition in a predator-prey environment~\cite{Dawkins1982}.
Both of these concepts can be described in nature as a concequence of the need for resources 
used in building survival machines, in Dawkin's words. 
By introducing the idea of depleting resources in organisms that are necessary for survival, 
i.e. \emph{energy}, we can guide their behaviour by placing objects inside the world which organisms can consume
in order to regain energy. The basic energy sources conceived by us are \emph{plants} and \emph{water}, which will
both be necessary for the survival of the organism.

The concept of water is simplified for our purposes as a static, lake-like body of water.
This resource is generated sparsely at central positions and will not be removed upon consumption.
Organisms are forced to interact with each other by the collectiv need to gather water at the same locations.

Plants are used to naturally control populations. We define them as a resource that is randomly generated
many times when the simulation is instantiated. These resources are removed upon consumption. In every tick, 
the plant has a chance determined by the plant spreading factor \(\psi \) to \emph{spread} by spawning a clone of itself
at a nearby position. Following these rules, a plant in an unpopulated part of the simulation will eventually spread 
into a forest.
This opens a resource-rich habitat in which organisms can thrive. Organisms can be expected to 
reproduce until the available plants start to dwindle. At this point, the organisms will starve until either 
the forrest has spread again or all the resources in the area have been consumed, resulting in either
migration or extinction. In any case, the number of organisms in the simulation will be bound by the amount of 
plants that can be consumed. 
\(\psi \) can be tuned according to the hardware limitations faced when running the simulation, as a higher value
will result in a lower upper bound on the number of organisms existing at the same time in the world.

The last kind of resource available to an organism is, as an alternative to plants, the consumption of 
other organisms. This allows for the formation of predator-prey dynamics which, 
because of the life/dinner principle~\cite{Dawkins1982}, put selective pressure on the prey to adapt to its 
environment.
