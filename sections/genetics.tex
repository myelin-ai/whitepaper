\section{Genetics}

As we have discovered in earlier work, combining complex neural networks 
with genetic algorithms can easily push the computational cost of advancement
beyond any reasonable limit~\cite{Ferner2017}.
To avoid this pitfall we can use the extra encoding potential provided by 
the addition of time as an input of our neural network's state~\cite{Paugam-Moisy2012}.

\subsection{Genome}
Our genome consists of Cluster \& Hox Genes.
These represent the information which is required to build the neural network.

\subsubsection{Important Terms}
\begin{description}
	\item[Cluster Genes] Cluster Genes act as a blueprint for a set of neurons and their connections, which we call a cluster.
	\item[Hox Genes] Hox Genes make up the building plan for the neural network,
	specifying how clusters should be placed and connected.
	\item[Placement Neuron] The neuron of the cluster that will be shared with the cluster that it is attached to. Instead of placing a new neuron, a neuron of an already placed cluster is being used as a substitute. See the \hyperref[itm:targetneuron]{Target Neuron} of the Hox Gene for more information.
	\item[Target Neuron\label{itm:targetneuron}] Refers to a neuron on a cluster that has already been placed.
\end{description}

\label{Example of a genome}
\subsection{Example of a genome}
We define three cluster genes and four hox genes.

\paragraph{Naming of Neurons}
Naming neurons uniquely without creating confusion is hard. We decided to use the $i$ prefix for cluster definitions (read $i_0$ as \emph{neuron with the index 0 on the cluster}).
Once a cluster is placed, we name the neurons in the order they have been placed in the network. As a prefix, we decided to use $n$. You can read $n_0$ as \emph{neuron with the index 0}.

\begin{figure}[H]
    \centering
    \subfile{graphics/first_cluster}
    \caption{First cluster - The placement neuron is $i_0$}
\end{figure}
\begin{figure}[H]
    \centering
    \subfile{graphics/second_cluster}
    \caption{Second cluster - The placement neuron is $i_0$}
\end{figure}
\begin{figure}[H]
    \centering
    \subfile{graphics/third_cluster}
    \caption{Third cluster - The placement neuron is $i_0$}
\end{figure}

\paragraph{The first hox gene} places a cluster based on the first cluster gene.
\FloatBarrier
\begin{figure}[H]
    \centering
    \subfile{graphics/first_cluster}
    \caption{Cluster placed by the first hox gene}
\end{figure}
\FloatBarrier

\newpage

\paragraph{The second hox gene} places clusters based on the second cluster gene on the clusters placed by the first hox gene.
The target neuron is $i_2$.
\begin{figure}[H]
    \centering
    \subfile{graphics/second_cluster_placed_on_first_hox}
    \caption{Cluster placed by the second hox gene}
\end{figure}

\paragraph{The third hox gene} places clusters based on the second cluster gene on the clusters placed by the second hox gene.
The target neuron is $i_1$.
\begin{figure}[H]
    \centering
    \subfile{graphics/second_cluster_placed_on_second_cluster}
    \caption{Cluster placed by the third hox gene}
\end{figure}

\paragraph{The fourth hox gene} places clusters based on the third cluster gene on all clusters based on the second cluster gene.
The target neuron is $i_1$.
\begin{figure}[H]
    \centering
    \subfile{graphics/third_cluster_placed_on_second_hox}
    \caption{Clusters placed by the fourth hox gene}
\end{figure}
