\section{Training}
The driving force behind evolution is the gene seeking immortality~\cite{Dawkins1976}.
As conventional training with backpropagation proves to be impractical when used with
spiking neural networks~\cite{Paugam-Moisy2012}, genetic algorithms seem like a natural training choice
when going for a nature-inspired simulation. Additionaly, it is our belief that the clear error
coefficient used in backpropagation is an unrealistic ideal and, by virtue of being calculated for
a specific task, contributes to the aforementioned lack of plasticity in second-generation neural networks.

As we want to train our AI to aquire general skills, the function governing the reproductive success of
the organisms for the genetic algorithm has to be general as well. The most general reward system
is one that does not reward anything specific at all. The organism that spreads its 
genes is not determined by a hard-coded reward system, but by the organism's ability to reproduce, 
just like in real life. This way, we shift the burden of defining what is considered ``good fitness''
away from a pre-defined algorithm into the fabric of the simulation itself.
For this, we propose giving the organisms the ability to consent to sexual reproduction.
